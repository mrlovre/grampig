\documentclass[basic]{grampig}

\title{に\textsuperscript{1}}
\pos{čestica}

\begin{document}
    \maketitle
    Označava primatelja radnje.\\
    Odgovara na pitanja \e{komu} ili \e{čemu}.
    \vspace{0.5em}

    \begin{table}
        \label{tab:tvorba}
        \centering
        \begin{tabular}{@{}ccccc@{}}
            \f{母}{はは} & + & \e{に} & $\Rightarrow$ & \e{\f{母}{はは}に} \bh
            majka & + &  & => & \e{majci} \\
        \end{tabular}
    \end{table}

    \begin{itemize}
        \item \f{手紙}{てがみ}を\e{\f{母}{はは}に}\f{書}{か}く。\bh
        Napisat ću pismo \e{majci}.
        \item \f{坂本}{さかもと}さんが\e{\f{猫}{ねこ}に}\f{餌}{えさ}を\f{上}{あ}げた。\bh
        Gđa.\ Sakamoto je \e{mački} dala hranu.
        \item \f{花子}{はなこ}さんがパリーから\f{絵葉書}{えはがき}を\f{一枚}{いちまい}も\e{\f{武}{たけし}くんに}\f{起}{おこ}こらなかった。\bh
        Hanako nije poslala ni jednu razglednicu iz Pariza \e{Takešiju}.
    \end{itemize}
\end{document}
